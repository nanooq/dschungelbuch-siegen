\section{Aufge(sc)hobener Kaffee}\index{Kaffee}

Basierend entstehend auf einer Italienischen Tradition entsteht das Prinzip von, „Suspended Coffee“.  Ein vorher bezahlter Kaffee wird aufgeschoben für eine Person, die Ihn sich selbst nicht leisten kann. Es ist ein simples Prinzip der Nächstenliebe: Trinkt man in einem der teilnehmenden Cafés einen Kaffee, bezahlt man nicht nur diesen, sondern einen zusätzlichen im voraus für jemanden, der sich selbst keinen leisten kann. Dieser „aufgeschobene Kaffee“ verbleibt im Café und eine Person, die weniger Geld zur Verfügung hat – egal aus welchem Grund – kann sich diesen dann kostenfrei abholen und genießen.\\
Internet: \href{http://www.aufgeschobener-kaffee-siegen.de}{http://www.aufgeschobener-kaffee-siegen.de}\\
Folgende machen mit:\\
\begin{itemize}
	\item Eiscafè Anna, Am Bahnhof 4, Siegen  
	\item Rosso Arancio, Eiscafe in der Citygalerie, Siegen
	\item Cafè Bienenstich, Am Bahnhof 4 - 12, Siegen
	\item Ciao, Pizza In-Biss, Alte Poststraße 19, Siegen
	\item Cafè Flocke, Marburger Straße 45. Siegen  
	\item Fünf 10, das Herz im TZ, ein Projekt der AWO, Birlenbacher Straße 18 Geisweid
	\item Cafè Planlos, Kohlbettstraße 18, Siegen  
	\item Pizzeria Uno, Fürst-Johann-Moritz-Straße 6, Siegen 
	\item Salz \& Pfeffer, Imbiss, Hessische Straße 8, Siegen
	\item Seelbacher Haarschmiede, Freudenberger Str. 476, Siegen Seelbach 
\end{itemize}