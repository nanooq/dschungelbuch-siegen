\section{Internationaler Bund - Schulsozialarbeit, Kinder- und Jugendtreff Siegen}
Der Kinder- und Jugendtreff K 52 bietet Schulsozialarbeit nach dem \enquote{Siegener Modell} sowie offene Kinder- und Jugendarbeit an. Auftraggeber ist die Stadt Siegen und das Angebot erfolgt in enger Kooperation mit der Stadt und den Schulen im Stadtteil. Die Einrichtung besteht seit November 2002.\\
\\
Schulsozialarbeit Siegen Kinder- und Jugendtreff\\
Frank Halberstadt\\
Heidenbergstr. 1c\\
57072 Siegen\\
Telefon: 02 29 5 - 24 28\\
E-Mail: \href{Frank.Halberstadt@internationaler-bund.de}{Frank.Halberstadt@internationaler-bund.de}\\
\\
Wesentliche Bestandteile der Angebote sind:
\begin{itemize}
	\item Arbeit mit Kindern: Die Kinder erfahren in ihrer sozialen, emotionalen und schulischen Entwicklung ganzheitliche Unterstützung.
	\item Arbeit mit Familien: In regelmäßig stattfindenden Elterngesprächen und Hausbesuchen wird unter Berücksichtigung der individuellen Möglichkeiten jeder Familie die Erziehungskompetenz der Eltern gestärkt.
	\item Kooperation mit der Schule: Wöchentlicher Austausch mit den Lehrern, Besuche im Unterricht und weitere gemeinsame Maßnahmen. ermöglichen, zwischen Schule und Elternhaus zu vermitteln und gemeinsam eine individuelle Förderplanung für die Kinder zu entwickeln.
	\item Zusammenarbeit mit anderen Einrichtungen: Die Einrichtung arbeitet zum Wohle des Kindes mit zahlreichen anderen am Heidenberg tätigen Einrichtungen und Fachdiensten zusammen.
	\item Angebote für Grundschüler
	\begin{itemize}
		\item Hausaufgabenbetreuung
		\item Einzelförderung
		\item Pädagogisch betreutes Spielzimmer
	\end{itemize}
	\item Offene Arbeit
	\begin{itemize}
		\item Kreativangebote
		\item Koch- und Backangebote
		\item Sport- / Schwimmangebote
		\item Medien AG: Projektarbeit (Theater etc.)
	\end{itemize}
\end{itemize}